\usepackage{amsmath,graphicx,latexsym,pslatex,hyperref,array,color}
\usepackage{tikz}
\usepackage[arrowmos]{circuitikz}
\usetikzlibrary{circuits.logic.US}
\usetikzlibrary{circuits.ee.IEC}
\usetikzlibrary{shapes}
\usetikzlibrary{shapes.geometric}
\usetikzlibrary{shapes.symbols}
\usetikzlibrary{positioning}
\usetikzlibrary{fit}
\usetikzlibrary{automata}
\usetikzlibrary{patterns}
\usetikzlibrary{chains}
\usetikzlibrary{calc}
\usetikzlibrary{decorations}
\usetikzlibrary{decorations.markings}
\usetikzlibrary{scopes}
\usetikzlibrary{shadings}
\input kvmacros.tex

\makeatletter

\pgfdeclareshape{dff}{
  % The 'minimum width' and 'minimum height' keys, not the content, determine
  % the size
  \savedanchor\northeast{%
    \pgfmathsetlength\pgf@x{\pgfshapeminwidth}%
    \pgfmathsetlength\pgf@y{\pgfshapeminheight}%
    \pgf@x=0.5\pgf@x
    \pgf@y=0.5\pgf@y
  }
  % This is redundant, but makes some things easier:
  \savedanchor\southwest{%
    \pgfmathsetlength\pgf@x{\pgfshapeminwidth}%
    \pgfmathsetlength\pgf@y{\pgfshapeminheight}%
    \pgf@x=-0.5\pgf@x
    \pgf@y=-0.5\pgf@y
  }
  % Inherit from rectangle
  \inheritanchorborder[from=rectangle]

  % Define same anchor a normal rectangle has
  \anchor{center}{\pgfpointorigin}
  \anchor{north}{\northeast \pgf@x=0pt}
  \anchor{east}{\northeast \pgf@y=0pt}
  \anchor{south}{\southwest \pgf@x=0pt}
  \anchor{west}{\southwest \pgf@y=0pt}
  \anchor{north east}{\northeast}
  \anchor{north west}{\northeast \pgf@x=-\pgf@x}
  \anchor{south west}{\southwest}
  \anchor{south east}{\southwest \pgf@x=-\pgf@x}
  \anchor{text}{
    \pgfpointorigin
    \advance\pgf@x by -.5\wd\pgfnodeparttextbox%
    \advance\pgf@y by -.5\ht\pgfnodeparttextbox%
    \advance\pgf@y by +.5\dp\pgfnodeparttextbox%
  }

  % Define anchors for signal ports
  \anchor{D}{
    \pgf@process{\northeast}%
    \pgf@x=-1\pgf@x%
    \pgf@y=.5\pgf@y%
  }
  \anchor{CLK}{
    \pgf@process{\northeast}%
    \pgf@x=-1\pgf@x%
    \pgf@y=-.66666\pgf@y%
  }
  \anchor{CE}{
    \pgf@process{\northeast}%
    \pgf@x=-1\pgf@x%
    \pgf@y=-0.33333\pgf@y%
  }
  \anchor{Q}{
    \pgf@process{\northeast}%
    \pgf@y=.5\pgf@y%
  }
  \anchor{Qn}{
    \pgf@process{\northeast}%
    \pgf@y=-.5\pgf@y%
  }
  \anchor{R}{
    \pgf@process{\northeast}%
    \pgf@x=0pt%
  }
  \anchor{S}{
    \pgf@process{\northeast}%
    \pgf@x=0pt%
    \pgf@y=-\pgf@y%
  }
  % Draw the rectangle box and the port labels
  \backgroundpath{
    % Rectangle box
    \pgfpathrectanglecorners{\southwest}{\northeast}
    % Angle (>) for clock input
    \pgf@anchor@dff@CLK
    \pgf@xa=\pgf@x \pgf@ya=\pgf@y
    \pgf@xb=\pgf@x \pgf@yb=\pgf@y
    \pgf@xc=\pgf@x \pgf@yc=\pgf@y
    \pgfmathsetlength\pgf@x{1.6ex} % size depends on font size
    \advance\pgf@ya by \pgf@x
    \advance\pgf@xb by \pgf@x
    \advance\pgf@yc by -\pgf@x
    \pgfpathmoveto{\pgfpoint{\pgf@xa}{\pgf@ya}}
    \pgfpathlineto{\pgfpoint{\pgf@xb}{\pgf@yb}}
    \pgfpathlineto{\pgfpoint{\pgf@xc}{\pgf@yc}}
    \pgfclosepath

    % Draw port labels
    \begingroup
    \tikzset{flip flop/port labels} % Use font from this style
    \tikz@textfont

    \pgf@anchor@dff@D
    \pgftext[left,base,at={\pgfpoint{\pgf@x}{\pgf@y}},x=\pgfshapeinnerxsep]{\raisebox{-0.75ex}{D}}

    \pgf@anchor@dff@CE
    \pgftext[left,base,at={\pgfpoint{\pgf@x}{\pgf@y}},x=\pgfshapeinnerxsep]{\raisebox{-0.75ex}{CE}}

    \pgf@anchor@dff@Q
    \pgftext[right,base,at={\pgfpoint{\pgf@x}{\pgf@y}},x=-\pgfshapeinnerxsep]{\raisebox{-.75ex}{Q}}

    \pgf@anchor@dff@Qn
    \pgftext[right,base,at={\pgfpoint{\pgf@x}{\pgf@y}},x=-\pgfshapeinnerxsep]{\raisebox{-.75ex}{$\overline{\mbox{Q}}$}}

    \pgf@anchor@dff@R
    \pgftext[top,at={\pgfpoint{\pgf@x}{\pgf@y}},y=-\pgfshapeinnerysep]{R}

    \pgf@anchor@dff@S
    \pgftext[bottom,at={\pgfpoint{\pgf@x}{\pgf@y}},y=\pgfshapeinnerysep]{S}
    \endgroup
  }
}

\pgfdeclareshape{dff}{
  % The 'minimum width' and 'minimum height' keys, not the content, determine
  % the size
  \savedanchor\northeast{%
    \pgfmathsetlength\pgf@x{\pgfshapeminwidth}%
    \pgfmathsetlength\pgf@y{\pgfshapeminheight}%
    \pgf@x=0.5\pgf@x
    \pgf@y=0.5\pgf@y
  }
  % This is redundant, but makes some things easier:
  \savedanchor\southwest{%
    \pgfmathsetlength\pgf@x{\pgfshapeminwidth}%
    \pgfmathsetlength\pgf@y{\pgfshapeminheight}%
    \pgf@x=-0.5\pgf@x
    \pgf@y=-0.5\pgf@y
  }
  % Inherit from rectangle
  \inheritanchorborder[from=rectangle]

  % Define same anchor a normal rectangle has
  \anchor{center}{\pgfpointorigin}
  \anchor{north}{\northeast \pgf@x=0pt}
  \anchor{east}{\northeast \pgf@y=0pt}
  \anchor{south}{\southwest \pgf@x=0pt}
  \anchor{west}{\southwest \pgf@y=0pt}
  \anchor{north east}{\northeast}
  \anchor{north west}{\northeast \pgf@x=-\pgf@x}
  \anchor{south west}{\southwest}
  \anchor{south east}{\southwest \pgf@x=-\pgf@x}
  \anchor{text}{
    \pgfpointorigin
    \advance\pgf@x by -.5\wd\pgfnodeparttextbox%
    \advance\pgf@y by -.5\ht\pgfnodeparttextbox%
    \advance\pgf@y by +.5\dp\pgfnodeparttextbox%
  }

  % Define anchors for signal ports
  \anchor{D}{
    \pgf@process{\northeast}%
    \pgf@x=-1\pgf@x%
    \pgf@y=.66\pgf@y%
  }
  \anchor{CLK}{
    \pgf@process{\northeast}%
    \pgf@x=-1\pgf@x%
    \pgf@y=-.66\pgf@y%
  }
  \anchor{CE}{
    \pgf@process{\northeast}%
    \pgf@x=-1\pgf@x%
    \pgf@y=-0.33333\pgf@y%
  }
  \anchor{Q}{
    \pgf@process{\northeast}%
    \pgf@y=.66\pgf@y%
  }
  \anchor{Qn}{
    \pgf@process{\northeast}%
    \pgf@y=-.66\pgf@y%
  }
  \anchor{R}{
    \pgf@process{\northeast}%
    \pgf@x=0pt%
  }
  \anchor{S}{
    \pgf@process{\northeast}%
    \pgf@x=0pt%
    \pgf@y=-\pgf@y%
  }
  % Draw the rectangle box and the port labels
  \backgroundpath{
    % Rectangle box
    \pgfpathrectanglecorners{\southwest}{\northeast}
    % Angle (>) for clock input
    \pgf@anchor@dff@CLK
    \pgf@xa=\pgf@x \pgf@ya=\pgf@y
    \pgf@xb=\pgf@x \pgf@yb=\pgf@y
    \pgf@xc=\pgf@x \pgf@yc=\pgf@y
    \pgfmathsetlength\pgf@x{1.2ex} % size depends on font size
    \advance\pgf@ya by \pgf@x
    \advance\pgf@xb by \pgf@x
    \advance\pgf@yc by -\pgf@x
    \pgfpathmoveto{\pgfpoint{\pgf@xa}{\pgf@ya}}
    \pgfpathlineto{\pgfpoint{\pgf@xb}{\pgf@yb}}
    \pgfpathlineto{\pgfpoint{\pgf@xc}{\pgf@yc}}
    \pgfclosepath

    % Draw port labels
    \begingroup
    \tikzset{flip flop/port labels} % Use font from this style
    \tikz@textfont

    \pgf@anchor@dff@D
    \pgftext[left,base,at={\pgfpoint{\pgf@x}{\pgf@y}},x=\pgfshapeinnerxsep]{\raisebox{-0.75ex}{D}}

%    \pgf@anchor@dff@CE
%    \pgftext[left,base,at={\pgfpoint{\pgf@x}{\pgf@y}},x=\pgfshapeinnerxsep]{\raisebox{-0.75ex}{CE}}

    \pgf@anchor@dff@Q
    \pgftext[right,base,at={\pgfpoint{\pgf@x}{\pgf@y}},x=-\pgfshapeinnerxsep]{\raisebox{-.75ex}{Q}}

    \pgf@anchor@dff@Qn
    \pgftext[right,base,at={\pgfpoint{\pgf@x}{\pgf@y}},x=-\pgfshapeinnerxsep]{\raisebox{-.75ex}{$\overline{\mbox{Q}}$}}

%    \pgf@anchor@dff@R
%    \pgftext[top,at={\pgfpoint{\pgf@x}{\pgf@y}},y=-\pgfshapeinnerysep]{R}

%    \pgf@anchor@dff@S
%    \pgftext[bottom,at={\pgfpoint{\pgf@x}{\pgf@y}},y=\pgfshapeinnerysep]{S}
    \endgroup
  }
}


% Key to add font macros to the current font
\tikzset{add font/.code={\expandafter\def\expandafter\tikz@textfont\expandafter{\tikz@textfont#1}}} 

% Define default style for this node
\tikzset{flip flop/port labels/.style={font=\sffamily\scriptsize}}
\tikzset{every dff node/.style={draw,minimum width=1.5cm,minimum 
height=2cm,thick,inner sep=1mm,outer sep=0pt,cap=round,add 
font=\sffamily}}

\pgfdeclareshape{jkff}{
  % The 'minimum width' and 'minimum height' keys, not the content, determine
  % the size
  \savedanchor\northeast{%
    \pgfmathsetlength\pgf@x{\pgfshapeminwidth}%
    \pgfmathsetlength\pgf@y{\pgfshapeminheight}%
    \pgf@x=0.5\pgf@x
    \pgf@y=0.5\pgf@y
  }
  % This is redundant, but makes some things easier:
  \savedanchor\southwest{%
    \pgfmathsetlength\pgf@x{\pgfshapeminwidth}%
    \pgfmathsetlength\pgf@y{\pgfshapeminheight}%
    \pgf@x=-0.5\pgf@x
    \pgf@y=-0.5\pgf@y
  }
  % Inherit from rectangle
  \inheritanchorborder[from=rectangle]

  % Define same anchor a normal rectangle has
  \anchor{center}{\pgfpointorigin}
  \anchor{north}{\northeast \pgf@x=0pt}
  \anchor{east}{\northeast \pgf@y=0pt}
  \anchor{south}{\southwest \pgf@x=0pt}
  \anchor{west}{\southwest \pgf@y=0pt}
  \anchor{north east}{\northeast}
  \anchor{north west}{\northeast \pgf@x=-\pgf@x}
  \anchor{south west}{\southwest}
  \anchor{south east}{\southwest \pgf@x=-\pgf@x}
  \anchor{text}{
    \pgfpointorigin
    \advance\pgf@x by -.5\wd\pgfnodeparttextbox%
    \advance\pgf@y by -.5\ht\pgfnodeparttextbox%
    \advance\pgf@y by +.5\dp\pgfnodeparttextbox%
  }

  % Define anchors for signal ports
  \anchor{J}{
    \pgf@process{\northeast}%
    \pgf@x=-1\pgf@x%
    \pgf@y=.66\pgf@y%
  }
  \anchor{CLK}{
    \pgf@process{\northeast}%
    \pgf@x=-1\pgf@x%
    \pgf@y=0\pgf@y%
  }
  \anchor{K}{
    \pgf@process{\northeast}%
    \pgf@x=-1\pgf@x%
    \pgf@y=-0.66\pgf@y%
  }
  \anchor{Q}{
    \pgf@process{\northeast}%
    \pgf@y=.66\pgf@y%
  }
  \anchor{Qn}{
    \pgf@process{\northeast}%
    \pgf@y=-.66\pgf@y%
  }
  \anchor{R}{
    \pgf@process{\northeast}%
    \pgf@x=0pt%
  }
  \anchor{S}{
    \pgf@process{\northeast}%
    \pgf@x=0pt%
    \pgf@y=-\pgf@y%
  }
  % Draw the rectangle box and the port labels
  \backgroundpath{
    % Rectangle box
    \pgfpathrectanglecorners{\southwest}{\northeast}
    % Angle (>) for clock input
    \pgf@anchor@jkff@CLK
    \pgf@xa=\pgf@x \pgf@ya=\pgf@y
    \pgf@xb=\pgf@x \pgf@yb=\pgf@y
    \pgf@xc=\pgf@x \pgf@yc=\pgf@y
    \pgfmathsetlength\pgf@x{1.2ex} % size depends on font size
    \advance\pgf@ya by \pgf@x
    \advance\pgf@xb by \pgf@x
    \advance\pgf@yc by -\pgf@x
    \pgfpathmoveto{\pgfpoint{\pgf@xa}{\pgf@ya}}
    \pgfpathlineto{\pgfpoint{\pgf@xb}{\pgf@yb}}
    \pgfpathlineto{\pgfpoint{\pgf@xc}{\pgf@yc}}
    \pgfclosepath

    % Draw port labels
    \begingroup
    \tikzset{flip flop/port labels} % Use font from this style
    \tikz@textfont

    \pgf@anchor@jkff@J
    \pgftext[left,base,at={\pgfpoint{\pgf@x}{\pgf@y}},x=\pgfshapeinnerxsep]{\raisebox{-0.75ex}{J}}

   \pgf@anchor@jkff@K
    \pgftext[left,base,at={\pgfpoint{\pgf@x}{\pgf@y}},x=\pgfshapeinnerxsep]{\raisebox{-0.75ex}{K}}

    \pgf@anchor@jkff@Q
    \pgftext[right,base,at={\pgfpoint{\pgf@x}{\pgf@y}},x=-\pgfshapeinnerxsep]{\raisebox{-.75ex}{Q}}

    \pgf@anchor@jkff@Qn
    \pgftext[right,base,at={\pgfpoint{\pgf@x}{\pgf@y}},x=-\pgfshapeinnerxsep]{\raisebox{-.75ex}{$\overline{\mbox{Q}}$}}

    \endgroup
  }
}

\tikzset{every jkff node/.style={draw,minimum width=1.5cm,minimum 
height=2cm,thick,inner sep=1mm,outer sep=0pt,cap=round,add 
font=\sffamily}}


\pgfdeclareshape{srlatch}{
  % The 'minimum width' and 'minimum height' keys, not the content, determine
  % the size
  \savedanchor\northeast{%
    \pgfmathsetlength\pgf@x{\pgfshapeminwidth}%
    \pgfmathsetlength\pgf@y{\pgfshapeminheight}%
    \pgf@x=0.5\pgf@x
    \pgf@y=0.5\pgf@y
  }
  % This is redundant, but makes some things easier:
  \savedanchor\southwest{%
    \pgfmathsetlength\pgf@x{\pgfshapeminwidth}%
    \pgfmathsetlength\pgf@y{\pgfshapeminheight}%
    \pgf@x=-0.5\pgf@x
    \pgf@y=-0.5\pgf@y
  }
  % Inherit from rectangle
  \inheritanchorborder[from=rectangle]

  % Define same anchor a normal rectangle has
  \anchor{center}{\pgfpointorigin}
  \anchor{north}{\northeast \pgf@x=0pt}
  \anchor{east}{\northeast \pgf@y=0pt}
  \anchor{south}{\southwest \pgf@x=0pt}
  \anchor{west}{\southwest \pgf@y=0pt}
  \anchor{north east}{\northeast}
  \anchor{north west}{\northeast \pgf@x=-\pgf@x}
  \anchor{south west}{\southwest}
  \anchor{south east}{\southwest \pgf@x=-\pgf@x}
  \anchor{text}{
    \pgfpointorigin
    \advance\pgf@x by -.5\wd\pgfnodeparttextbox%
    \advance\pgf@y by -.5\ht\pgfnodeparttextbox%
    \advance\pgf@y by +.5\dp\pgfnodeparttextbox%
  }

  % Define anchors for signal ports
  \anchor{S}{
    \pgf@process{\northeast}%
    \pgf@x=-1\pgf@x%
    \pgf@y=.66\pgf@y%
  }
  \anchor{R}{
    \pgf@process{\northeast}%
    \pgf@x=-1\pgf@x%
    \pgf@y=-.66\pgf@y%
  }
  \anchor{CE}{
    \pgf@process{\northeast}%
    \pgf@x=-1\pgf@x%
    \pgf@y=-0.33333\pgf@y%
  }
  \anchor{Q}{
    \pgf@process{\northeast}%
    \pgf@y=.66\pgf@y%
  }
  \anchor{Qn}{
    \pgf@process{\northeast}%
    \pgf@y=-.66\pgf@y%
  }
% %  \anchor{R}{
%     \pgf@process{\northeast}%
%     \pgf@x=0pt%
%   }
%   \anchor{S}{
%     \pgf@process{\northeast}%
%     \pgf@x=0pt%
%     \pgf@y=-\pgf@y%
%   }
  % Draw the rectangle box and the port labels
  \backgroundpath{
    % Rectangle box
    \pgfpathrectanglecorners{\southwest}{\northeast}
    % Angle (>) for clock input


    % Draw port labels
    \begingroup
    \tikzset{flip flop/port labels} % Use font from this style
    \tikz@textfont

    \pgf@anchor@srlatch@S
    \pgftext[left,base,at={\pgfpoint{\pgf@x}{\pgf@y}},x=\pgfshapeinnerxsep]{\raisebox{-0.75ex}{S}}

    \pgf@anchor@srlatch@R
    \pgftext[left,base,at={\pgfpoint{\pgf@x}{\pgf@y}},x=\pgfshapeinnerxsep]{\raisebox{-0.75ex}{R}}

    \pgf@anchor@srlatch@Q
    \pgftext[right,base,at={\pgfpoint{\pgf@x}{\pgf@y}},x=-\pgfshapeinnerxsep]{\raisebox{-.75ex}{Q}}

    \pgf@anchor@srlatch@Qn
    \pgftext[right,base,at={\pgfpoint{\pgf@x}{\pgf@y}},x=-\pgfshapeinnerxsep]{\raisebox{-.75ex}{$\overline{\mbox{Q}}$}}

%    \pgf@anchor@srlatch@R
%    \pgftext[top,at={\pgfpoint{\pgf@x}{\pgf@y}},y=-\pgfshapeinnerysep]{R}

%    \pgf@anchor@srlatch@S
%    \pgftext[bottom,at={\pgfpoint{\pgf@x}{\pgf@y}},y=\pgfshapeinnerysep]{S}
    \endgroup
  }
}

\tikzset{every srlatch node/.style={draw,minimum width=1.5cm,minimum 
height=2cm,thick,inner sep=1mm,outer sep=0pt,cap=round,add 
font=\sffamily}}

% xor gate multiple inputs

\pgfdeclareshape{xor gate US}{%
	\expandafter\pgfutil@g@addto@macro\csname pgf@sh@s@xor gate US\endcsname{%
		\pgf@lib@sh@logicgate@parseinputs{1024}% 
		%
		\pgfmathloop%
		\ifnum\pgfmathcounter>\pgf@lib@sh@logicgate@numinputs%
		\else%
			\pgfutil@ifundefined{pgf@anchor@xor gate US@input \pgfmathcounter}{%
				\expandafter\xdef\csname pgf@anchor@xor gate US@input \pgfmathcounter\endcsname{%
					\noexpand\pgf@lib@sh@logicgate@XOR@inputanchor{\pgfmathcounter}%
				}%
			}{}%
		\repeatpgfmathloop%
		\ifnum\pgf@lib@sh@logicgate@numinputs<2\relax%
			\PackageError{PGF}{An xor gate must have at two inputs}{}%
		\fi%
	}%
	\savedmacro\numinputs{\let\numinputs\pgf@lib@sh@logicgate@numinputs}%
	\saveddimen\invertedradius{%
		\pgfmathsetlength\pgf@x{\pgfkeysvalueof{/pgf/logic gate inverted radius}}%
	}%
	\saveddimen\halflinewidth{%
		\pgf@x.5\pgflinewidth%
	}%
	\saveddimen\outerinvertedradius{%
		\pgfmathsetlength\pgf@x{\pgfkeysvalueof{/pgf/logic gate inverted radius}}%
		\advance\pgf@x.5\pgflinewidth%
	}
	\savedmacro\dimensions{\pgf@lib@sh@logicgates@dimensions@orUS}
	\savedanchor\centerpoint{%
		\pgf@x.5\wd\pgfnodeparttextbox%
		\pgf@y.5\ht\pgfnodeparttextbox%
		\advance\pgf@y-.5\dp\pgfnodeparttextbox%
	}
	\savedanchor\midpoint{%
		\pgf@x.5\wd\pgfnodeparttextbox%
		\pgfmathsetlength\pgf@y{+0.5ex}%
	}
	\savedanchor\basepoint{%
		\pgf@x.5\wd\pgfnodeparttextbox%
		\pgf@y0pt%
	}
	\anchor{center}{\centerpoint}%
	\anchor{mid}{\midpoint}%
	\inheritanchor[from=or gate US]{mid east}
	\anchor{mid west}{%
		\csname pgf@anchor@xor gate US@north west\endcsname%
		\pgf@xa\pgf@x%
		\midpoint%
		\pgf@x\pgf@xa}
	\anchor{base}{\basepoint}%
	\inheritanchor[from=or gate US]{base east}
	\anchor{base west}{%
		\csname pgf@anchor@xor gate US@north west\endcsname%
		\pgf@xa\pgf@x%
		\basepoint%
		\pgf@x\pgf@xa}
	\inheritanchor[from=or gate US]{base}
	\inheritanchor[from=or gate US]{output}
	\inheritanchor[from=or gate US]{east}
	\inheritanchor[from=or gate US]{north east}
	\inheritanchor[from=or gate US]{south east}
	\inheritanchor[from=or gate US]{north}
	\inheritanchor[from=or gate US]{south}
	\anchor{south west}{%
		\dimensions%
		\pgf@xa\halfside%
		\pgf@xa-3.232051\pgf@xa% (7/6 + 2*cos(30) + 1/3) * x
		\pgf@xb\halfside%
		\pgf@xb2.0\pgf@xb%
		\advance\pgf@xb-\halflinewidth%
		\advance\pgf@xa.866025\pgf@xb%
		\pgf@ya.5\pgf@xb%
		\centerpoint%
		\advance\pgf@x\pgf@xa%
		\ifpgfgateanchorsuseboundingrectangle%
			\advance\pgf@y-\halfheight%
		\else%
			\advance\pgf@y-\pgf@ya%
		\fi%
	}
	\anchor{north west}{%
		\dimensions%
		\pgf@xa\halfside%
		\pgf@xa-3.232051\pgf@xa% (7/6 + 2*cos(30) + 1/3) * x
		\pgf@xb\halfside%
		\pgf@xb2.0\pgf@xb%
		\advance\pgf@xb-\halflinewidth%
		\advance\pgf@xa.866025\pgf@xb%
		\pgf@ya.5\pgf@xb%
		\centerpoint%
		\advance\pgf@x\pgf@xa%
		\ifpgfgateanchorsuseboundingrectangle%
			\advance\pgf@y\halfheight%
		\else%
			\advance\pgf@y\pgf@ya%
		\fi%
	}
	\anchor{west}{%
		\dimensions%
		\pgf@ya\halfside%
		\pgf@yb2.0\pgf@ya%
		%
		\pgf@xb\pgf@yb%
		\advance\pgf@yb-\halflinewidth%
		\pgfmathdivide@{0}{\pgfmath@tonumber{\pgf@yb}}%
		\pgfmathasin@{\pgfmathresult}%
		\pgfmathcos@{\pgfmathresult}%
		%
		\pgf@xc-1.166666\pgf@ya%
		\advance\pgf@xc-.866025\pgf@xb%
		\advance\pgf@xc\pgfmathresult\pgf@yb%
		\advance\pgf@xc\halflinewidth%
		\advance\pgf@xc-\outerxsep%
		%
		\centerpoint%
		\advance\pgf@x\pgf@xc%
		\pgf@xa\halfside%
		\advance\pgf@x-.333333\pgf@xa%
		\ifpgfgateanchorsuseboundingrectangle%
			\pgf@xa2.0\pgf@xa%
			\advance\pgf@x-0.133974\pgf@xa%
		\fi%
	}
	\backgroundpath{%
		\dimensions%
		\pgf@xc\halfwidth%
		\pgf@yc\halfheight%
		\advance\pgf@xc-\outerxsep%
		\advance\pgf@yc-\outerysep%
		{%
			\pgftransformshift{\centerpoint}%
			\pgfpathmoveto{\pgfqpoint{-.16666\pgf@xc}{\pgf@yc}}%
			{%
				\pgf@yc2.0\pgf@yc%
				\edef\pgf@marshal{%
					\noexpand\pgfpatharc{90}{30}{\the\pgf@yc}%
				}%
				\pgf@marshal%
			}
			{%
				\pgf@yc2.0\pgf@yc%
				\edef\pgf@marshal{%
					\noexpand\pgfpatharc{-30}{-90}{\the\pgf@yc}%
				}%
				\pgf@marshal%
			}
			\pgfpathlineto{\pgfqpoint{-1.16666\pgf@xc}{-\pgf@yc}}%
			{%
				\pgf@yc2.66666\pgf@yc%
				\pgfpatharc{-22}{0}{+1.166666\pgf@yc and +\pgf@yc}%			
			}
			{%
				\pgf@yc2.66666\pgf@yc%
				\pgfpatharc{0}{22}{+1.166666\pgf@yc and +\pgf@yc}%			
			}%
			\pgfpathclose%
			%
			% Draw the inputs.
			%
			\pgfutil@tempdima2.0\pgf@yc%
			\c@pgf@counta\numinputs%
			\advance\c@pgf@counta1\relax%
			\divide\pgfutil@tempdima\c@pgf@counta%
			\pgfmathloop%
			\ifnum\pgfmathcounter>\numinputs%
			\else%
				\advance\pgf@yc-\pgfutil@tempdima%
				\expandafter\ifx\expandafter\pgf@lib@sh@itext\csname input-\pgfmathcounter\endcsname%
					{%
						\pgfpathcircle{%
							\pgf@ya\halfside%
							\pgf@yb2.0\pgf@ya%
							\pgf@xa\pgf@yb%
							\advance\pgf@yb-\halflinewidth%
							\pgfmathdivide@{\pgfmath@tonumber{\pgf@yc}}{\pgfmath@tonumber{\pgf@yb}}%
							\pgfmathasin@{\pgfmathresult}%
							\pgfmathcos@{\pgfmathresult}%
							%
							\pgf@x-1.5\pgf@ya%
							\advance\pgf@x-.866025\pgf@xa%
							\advance\pgf@x\pgfmathresult\pgf@yb%
							\advance\pgf@x-\invertedradius%
							\pgf@y\pgf@yc%
						}{+\invertedradius}%				
					}%
				\fi%				
			\repeatpgfmathloop%
			%
			% Now, some fooling around to stop the `tail' being filled.
			% Technically it still is, but it isn't visible.
			%
			\pgf@xc\halfside%
			\pgf@yc\halfside%
			\pgfpathmoveto{\pgfqpoint{-1.5\pgf@xc}{-\pgf@yc}}%
			\pgf@yc2.0\pgf@yc%
			\pgfmathloop%
			\ifnum\pgfmathcounter<61\relax%
				{%
					\pgfextract@process\point{%
						\pgfpointadd{%
							\pgf@x\halfside%
							\pgf@x-3.232051\pgf@x% (7/6 + 2*cos(30) + 1/3) * x
							\pgf@y0pt%
						}{%
							\pgfpointpolar{\pgfmathcounter-30}{+\pgf@yc}%
						}%
					}%
					\pgfpathlineto{\point}%
					\pgfpathmoveto{\point}%
				}
			\repeatpgfmathloop%
		}%
	}%
	\anchorborder{%
		\pgfextract@process\externalpoint{}%
		\pgfextract@process\externalpoint{\pgfpointadd{\centerpoint}{\externalpoint}}%
		\pgf@xa\pgf@x%
		\pgf@ya\pgf@y%
		\centerpoint%
		\pgf@xb\pgf@x%
		\pgf@yb\pgf@y%
		\pgfmathanglebetweenpoints{\centerpoint}{\externalpoint}%
		\let\externalangle\pgfmathresult%
		\dimensions%
		\pgf@xc\halfside%
		%
		\pgf@xc-.166666\pgf@xc%
		\ifdim\pgf@xa<\pgf@xc%
			\pgfmathanglebetweenpoints{\centerpoint}%
			{%
				\pgfgateanchorsuseboundingrectangletrue%
				\csname pgf@anchor@xor gate US@north west\endcsname%
			}%
			\ifdim\externalangle pt<\pgfmathresult pt\relax%
				\pgfpointintersectionoflines{\externalpoint}{\centerpoint}%
					{%
						\pgfgateanchorsuseboundingrectangletrue%
						\csname pgf@anchor@xor gate US@north\endcsname%
					}%
					{%
						\pgfgateanchorsuseboundingrectangletrue%
						\csname pgf@anchor@xor gate US@north west\endcsname%
					}%
			\else%
				\pgfmathsubtract@{360}{\pgfmathresult}%
				\ifdim\externalangle pt>\pgfmathresult pt\relax%
					\pgfpointintersectionoflines{\externalpoint}{\centerpoint}%
					{%
						\pgfgateanchorsuseboundingrectangletrue%
						\csname pgf@anchor@xor gate US@south\endcsname%
					}%
					{%
						\pgfgateanchorsuseboundingrectangletrue%
						\csname pgf@anchor@xor gate US@south west\endcsname%
					}%
				\else%
					\ifdim\pgf@ya>\pgf@yb%
						\pgf@yc\halfheight%
						\advance\pgf@yc\halfside%
						\advance\pgf@yc-\outerxsep%
						\pgf@process{%
							\pgfmathpointintersectionoflineandarc{\centerpoint}{\externalpoint}%
							{%
								\centerpoint%
								\pgf@xa\halfside%
								\advance\pgf@x-1.166666\pgf@xa%
								\pgf@xa2.0\pgf@xa%
								\advance\pgf@x-.866025\pgf@xa%
								\advance\pgf@x-\outerxsep%
								\advance\pgf@x-.166666\pgf@xa%
							}%
							{0}{90}{+\pgf@yc}%
						}%
					\else%
						\pgf@yc\halfheight%
						\advance\pgf@yc\halfside%
						\advance\pgf@yc-\outerxsep%
						\pgf@process{%
							\pgfmathpointintersectionoflineandarc{\centerpoint}{\externalpoint}%
							{%
								\centerpoint%
								\pgf@xa\halfside%
								\advance\pgf@x-1.166666\pgf@xa%
								\pgf@xa2.0\pgf@xa%
								\advance\pgf@x-.866025\pgf@xa%
								\advance\pgf@x-\outerxsep%
								\advance\pgf@x-.166666\pgf@xa%
							}%
							{270}{360}{+\pgf@yc}%
						}%
					\fi%
				\fi%
			\fi%
		\else%
			\ifdim\pgf@y=0pt\relax%
				\csname pgf@anchor@and gate US@east\endcsname%
			\else%
				\pgf@xc\halfwidth%
				\advance\pgf@xc\halfside%
				\pgf@yc\halfheight%
				\advance\pgf@yc\halfside%
				\pgf@xb\halfside%
				\pgf@xb-.166666\pgf@xb%
				\ifdim\pgf@ya<0pt%
					\pgfmathpointintersectionoflineandarc{\externalpoint}{\centerpoint}%
					{%
						\centerpoint%
						\advance\pgf@y\halfside%	
						\advance\pgf@x\pgf@xb%		
					}%
					{270}{330}{+\pgf@yc}%
				\else%
					\pgfmathpointintersectionoflineandarc{\externalpoint}{\centerpoint}%
					{%
						\centerpoint%
						\advance\pgf@y-\halfside%	
						\advance\pgf@x\pgf@xb%
					}%
					{30}{90}{+\pgf@xc and +\pgf@yc}%
				\fi%
			\fi%
		\fi%
	}%
}


% xnor gate - multiple inputs

\pgfdeclareshape{xnor gate US}{%
	\expandafter\pgfutil@g@addto@macro\csname pgf@sh@s@xnor gate US\endcsname{%
		\pgf@lib@sh@logicgate@parseinputs{1024}% 
		%
		\pgfmathloop%
		\ifnum\pgfmathcounter>\pgf@lib@sh@logicgate@numinputs%
		\else%
			\pgfutil@ifundefined{pgf@anchor@xnor gate US@input \pgfmathcounter}{%
				\expandafter\xdef\csname pgf@anchor@xnor gate US@input \pgfmathcounter\endcsname{%
					\noexpand\pgf@lib@sh@logicgate@XOR@inputanchor{\pgfmathcounter}%
				}%
			}{}%
		\repeatpgfmathloop%
		\ifnum\pgf@lib@sh@logicgate@numinputs<2\relax%
			\PackageError{PGF}{An xnor gate must have two inputs}{}%
		\fi%
	}%
	\savedmacro\numinputs{\let\numinputs\pgf@lib@sh@logicgate@numinputs}%
	\saveddimen\invertedradius{%
		\pgfmathsetlength\pgf@x{\pgfkeysvalueof{/pgf/logic gate inverted radius}}%
	}%
	\saveddimen\halflinewidth{%
		\pgf@x.5\pgflinewidth%
	}%
	\saveddimen\outerinvertedradius{%
		\pgfmathsetlength\pgf@x{\pgfkeysvalueof{/pgf/logic gate inverted radius}}%
		\advance\pgf@x.5\pgflinewidth%
	}
	\savedmacro\dimensions{\pgf@lib@sh@logicgates@dimensions@orUS}
	\savedanchor\centerpoint{%
		\pgf@x.5\wd\pgfnodeparttextbox%
		\pgf@y.5\ht\pgfnodeparttextbox%
		\advance\pgf@y-.5\dp\pgfnodeparttextbox%
	}
	\savedanchor\midpoint{%
		\pgf@x.5\wd\pgfnodeparttextbox%
		\pgfmathsetlength\pgf@y{+0.5ex}%
	}
	\savedanchor\basepoint{%
		\pgf@x.5\wd\pgfnodeparttextbox%
		\pgf@y0pt%
	}
	\anchor{center}{\centerpoint}%
	\anchor{mid}{\midpoint}%
	\inheritanchor[from=xor gate US]{mid east}
	\inheritanchor[from=xor gate US]{mid west}
	\anchor{base}{\basepoint}%
	\inheritanchor[from=xor gate US]{base east}
	\inheritanchor[from=xor gate US]{base west}
	\inheritanchor[from=xor gate US]{base}
	\inheritanchor[from=nor gate US]{output}
	\inheritanchor[from=xor gate US]{east}
	\inheritanchor[from=xor gate US]{north east}
	\inheritanchor[from=xor gate US]{south east}
	\inheritanchor[from=xor gate US]{north}
	\inheritanchor[from=xor gate US]{south}
	\inheritanchor[from=xor gate US]{south west}
	\inheritanchor[from=xor gate US]{north west}
	\inheritanchor[from=xor gate US]{west}
	\backgroundpath{%
		\dimensions%
		\pgf@xc\halfwidth%
		\pgf@yc\halfheight%
		\advance\pgf@xc-\outerxsep%
		\advance\pgf@yc-\outerysep%
		{%
			\pgftransformshift{\centerpoint}%
			\pgfpathmoveto{\pgfqpoint{-.16666\pgf@xc}{\pgf@yc}}%
			{%
				\pgf@yc2.0\pgf@yc%
				\edef\pgf@marshal{%
					\noexpand\pgfpatharc{90}{30}{\the\pgf@yc}%
				}%
				\pgf@marshal%
			}
			{%
				\pgf@yc2.0\pgf@yc%
				\edef\pgf@marshal{%
					\noexpand\pgfpatharc{-30}{-90}{\the\pgf@yc}%
				}%
				\pgf@marshal%
			}
			\pgfpathlineto{\pgfqpoint{-1.16666\pgf@xc}{-\pgf@yc}}%
			{%
				\pgf@yc2.66666\pgf@yc%
				\pgfpatharc{-22}{0}{+1.166666\pgf@yc and +\pgf@yc}%			
			}
			{%
				\pgf@yc2.66666\pgf@yc%
				\pgfpatharc{0}{22}{+1.166666\pgf@yc and +\pgf@yc}%			
			}%
			\pgfpathclose%
			%
			% Draw the output inverter.
			%
			{%
				\pgfpathcircle{%
					\pgf@x-.166666\pgf@xc%
					\pgf@yc2.0\pgf@yc%
					\advance\pgf@x.866025\pgf@yc%
					\advance\pgf@x\outerinvertedradius%
					\pgf@y0pt%
				}{+\invertedradius}%			
			}%
			%
			% Draw the inputs.
			%
			\pgfutil@tempdima2.0\pgf@yc%
			\c@pgf@counta\numinputs%
			\advance\c@pgf@counta1\relax%
			\divide\pgfutil@tempdima\c@pgf@counta%
			\pgfmathloop%
			\ifnum\pgfmathcounter>\numinputs%
			\else%
				\advance\pgf@yc-\pgfutil@tempdima%
				\expandafter\ifx\expandafter\pgf@lib@sh@itext\csname input-\pgfmathcounter\endcsname%
					{%
						\pgfpathcircle{%
							\pgf@ya\halfside%
							\pgf@yb2.0\pgf@ya%
							\pgf@xa\pgf@yb%
							\advance\pgf@yb-\halflinewidth%
							\pgfmathdivide@{\pgfmath@tonumber{\pgf@yc}}{\pgfmath@tonumber{\pgf@yb}}%
							\pgfmathasin@{\pgfmathresult}%
							\pgfmathcos@{\pgfmathresult}%
							%
							\pgf@x-1.5\pgf@ya%
							\advance\pgf@x-.866025\pgf@xa%
							\advance\pgf@x\pgfmathresult\pgf@yb%
							\advance\pgf@x-\invertedradius%
							\pgf@y\pgf@yc%
						}{+\invertedradius}%				
					}%
				\fi%				
			\repeatpgfmathloop%
			%
			% Now, some fooling around to stop the `tail' being filled.
			%
			\pgf@xc\halfside%
			\pgf@yc\halfside%
			\pgfpathmoveto{\pgfqpoint{-1.5\pgf@xc}{-\pgf@yc}}%
			\pgf@yc2.0\pgf@yc%
			\pgfmathloop%
			\ifnum\pgfmathcounter<61\relax%
				{%
					\pgfextract@process\point{%
						\pgfpointadd{%
							\pgf@x\halfside%
							\pgf@x-3.232051\pgf@x% (7/6 + 2*cos(30) + 1/3) * x
							\pgf@y0pt%
						}{%
							\pgfpointpolar{\pgfmathcounter-30}{+\pgf@yc}%
						}%
					}%
					\pgfpathlineto{\point}%
					\pgfpathmoveto{\point}%
				}
			\repeatpgfmathloop%
		}%
	}%
	\inheritanchorborder[from=xor gate US]
}

\pgfdeclareshape{mux81}{
  % The 'minimum width' and 'minimum height' keys, not the content, determine
  % the size
  \savedanchor\northeast{%
    \pgfmathsetlength\pgf@x{\pgfshapeminwidth}%
    \pgfmathsetlength\pgf@y{\pgfshapeminheight}%
    \pgf@x=0.5\pgf@x
    \pgf@y=0.5\pgf@y
  }
  % This is redundant, but makes some things easier:
  \savedanchor\southwest{%
    \pgfmathsetlength\pgf@x{\pgfshapeminwidth}%
    \pgfmathsetlength\pgf@y{\pgfshapeminheight}%
    \pgf@x=-0.5\pgf@x
    \pgf@y=-0.5\pgf@y
  }
  % Inherit from rectangle
  \inheritanchorborder[from=rectangle]

  % Define same anchor a normal rectangle has
  \anchor{center}{\pgfpointorigin}
  \anchor{north}{\northeast \pgf@x=0pt}
  \anchor{east}{\northeast \pgf@y=0pt}
  \anchor{south}{\southwest \pgf@x=0pt}
  \anchor{west}{\southwest \pgf@y=0pt}
  \anchor{north east}{\northeast}
  \anchor{north west}{\northeast \pgf@x=-\pgf@x}
  \anchor{south west}{\southwest}
  \anchor{south east}{\southwest \pgf@x=-\pgf@x}
  \anchor{text}{
    \pgfpointorigin
    \advance\pgf@x by -.5\wd\pgfnodeparttextbox%
    \advance\pgf@y by -.5\ht\pgfnodeparttextbox%
    \advance\pgf@y by +.5\dp\pgfnodeparttextbox%
  }

  % Define anchors for signal ports
  \anchor{input 7}{
    \pgf@process{\northeast}%
    \pgf@x=-1\pgf@x%
    \pgf@y=.87\pgf@y%
  }
  \anchor{input 6}{
    \pgf@process{\northeast}%
    \pgf@x=-1\pgf@x%
    \pgf@y=.66\pgf@y%
  }
  \anchor{input 5}{
    \pgf@process{\northeast}%
    \pgf@x=-1\pgf@x%
    \pgf@y=.45\pgf@y%
  }
  \anchor{input 4}{
    \pgf@process{\northeast}%
    \pgf@x=-1\pgf@x%
    \pgf@y=0.24\pgf@y%
  }
  \anchor{input 3}{
    \pgf@process{\northeast}%
    \pgf@x=-1\pgf@x%
    \pgf@y=0.03\pgf@y%
  }
  \anchor{input 2}{
    \pgf@process{\northeast}%
    \pgf@x=-1\pgf@x%
    \pgf@y=-0.18\pgf@y%
  }
  \anchor{input 1}{
    \pgf@process{\northeast}%
    \pgf@x=-1\pgf@x%
    \pgf@y=-0.39\pgf@y%
  }
  \anchor{input 0}{
    \pgf@process{\northeast}%
    \pgf@x=-1\pgf@x%
    \pgf@y=-0.6\pgf@y%
  }
  \anchor{S 2}{
    \pgf@process{\northeast}%
    \pgf@x=-0.25\pgf@x%
    \pgf@y=-1\pgf@y%
  }
  \anchor{S 1}{
    \pgf@process{\northeast}%
    \pgf@x=0.25\pgf@x%
    \pgf@y=-1\pgf@y%
  }
  \anchor{S 0}{
    \pgf@process{\northeast}%
    \pgf@x=0.75\pgf@x%
    \pgf@y=-1\pgf@y%
  }

  \anchor{output}{
   \pgf@process{\northeast}%
    \pgf@y=0.5\pgf@y%
  }

  % Draw the rectangle box and the port labels
  \backgroundpath{
    % Rectangle box
    \pgfpathrectanglecorners{\southwest}{\northeast}
 

    % Draw port labels
    \begingroup
    \tikzset{flip flop/port labels} % Use font from this style
    \tikz@textfont

    @\csname pgf@anchor@mux81@input 7\endcsname
    \pgftext[left,base,at={\pgfpoint{\pgf@x}{\pgf@y}},x=\pgfshapeinnerxsep]{\raisebox{-0.75ex}{111}}

    @\csname pgf@anchor@mux81@input 6\endcsname
    \pgftext[left,base,at={\pgfpoint{\pgf@x}{\pgf@y}},x=\pgfshapeinnerxsep]{\raisebox{-0.75ex}{110}}

    @\csname pgf@anchor@mux81@input 5\endcsname
    \pgftext[left,base,at={\pgfpoint{\pgf@x}{\pgf@y}},x=\pgfshapeinnerxsep]{\raisebox{-0.75ex}{101}}

    @\csname pgf@anchor@mux81@input 4\endcsname
    \pgftext[left,base,at={\pgfpoint{\pgf@x}{\pgf@y}},x=\pgfshapeinnerxsep]{\raisebox{-0.75ex}{100}}

    @\csname pgf@anchor@mux81@input 3\endcsname
    \pgftext[left,base,at={\pgfpoint{\pgf@x}{\pgf@y}},x=\pgfshapeinnerxsep]{\raisebox{-0.75ex}{011}}

    @\csname pgf@anchor@mux81@input 2\endcsname
    \pgftext[left,base,at={\pgfpoint{\pgf@x}{\pgf@y}},x=\pgfshapeinnerxsep]{\raisebox{-0.75ex}{010}}

    @\csname pgf@anchor@mux81@input 1\endcsname
    \pgftext[left,base,at={\pgfpoint{\pgf@x}{\pgf@y}},x=\pgfshapeinnerxsep]{\raisebox{-0.75ex}{001}}

    @\csname pgf@anchor@mux81@input 0\endcsname
    \pgftext[left,base,at={\pgfpoint{\pgf@x}{\pgf@y}},x=\pgfshapeinnerxsep]{\raisebox{-0.75ex}{000}}

    @\csname pgf@anchor@mux81@S 0\endcsname
    \pgftext[bottom,at={\pgfpoint{\pgf@x}{\pgf@y}},y=\pgfshapeinnerysep]{S$_0$}

    @\csname pgf@anchor@mux81@S 1\endcsname
    \pgftext[bottom,at={\pgfpoint{\pgf@x}{\pgf@y}},y=\pgfshapeinnerysep]{S$_1$}

    @\csname pgf@anchor@mux81@S 2\endcsname
    \pgftext[bottom,at={\pgfpoint{\pgf@x}{\pgf@y}},y=\pgfshapeinnerysep]{S$_2$}

    @\csname pgf@anchor@mux81@output\endcsname
    \pgftext[right,base,at={\pgfpoint{\pgf@x}{\pgf@y}},x=-\pgfshapeinnerxsep]{\raisebox{-.75ex}{Z}}

    \endgroup
  }
}

\tikzset{every mux81 node/.style={draw,minimum width=2cm,minimum 
height=3cm,inner sep=1mm,outer sep=0pt,cap=round,add font=\sffamily}}

\pgfdeclareshape{mux41}{
  % The 'minimum width' and 'minimum height' keys, not the content, determine
  % the size
  \savedanchor\northeast{%
    \pgfmathsetlength\pgf@x{\pgfshapeminwidth}%
    \pgfmathsetlength\pgf@y{\pgfshapeminheight}%
    \pgf@x=0.5\pgf@x
    \pgf@y=0.5\pgf@y
  }
  % This is redundant, but makes some things easier:
  \savedanchor\southwest{%
    \pgfmathsetlength\pgf@x{\pgfshapeminwidth}%
    \pgfmathsetlength\pgf@y{\pgfshapeminheight}%
    \pgf@x=-0.5\pgf@x
    \pgf@y=-0.5\pgf@y
  }
  % Inherit from rectangle
  \inheritanchorborder[from=rectangle]

  % Define same anchor a normal rectangle has
  \anchor{center}{\pgfpointorigin}
  \anchor{north}{\northeast \pgf@x=0pt}
  \anchor{east}{\northeast \pgf@y=0pt}
  \anchor{south}{\southwest \pgf@x=0pt}
  \anchor{west}{\southwest \pgf@y=0pt}
  \anchor{north east}{\northeast}
  \anchor{north west}{\northeast \pgf@x=-\pgf@x}
  \anchor{south west}{\southwest}
  \anchor{south east}{\southwest \pgf@x=-\pgf@x}
  \anchor{text}{
    \pgfpointorigin
    \advance\pgf@x by -.5\wd\pgfnodeparttextbox%
    \advance\pgf@y by -.5\ht\pgfnodeparttextbox%
    \advance\pgf@y by +.5\dp\pgfnodeparttextbox%
  }

  % Define anchors for signal ports
  \anchor{input 3}{
    \pgf@process{\northeast}%
    \pgf@x=-1\pgf@x%
    \pgf@y=0.8\pgf@y%
  }
  \anchor{input 2}{
    \pgf@process{\northeast}%
    \pgf@x=-1\pgf@x%
    \pgf@y=0.4\pgf@y%
  }
  \anchor{input 1}{
    \pgf@process{\northeast}%
    \pgf@x=-1\pgf@x%
    \pgf@y=-0.0\pgf@y%
  }
  \anchor{input 0}{
    \pgf@process{\northeast}%
    \pgf@x=-1\pgf@x%
    \pgf@y=-0.4\pgf@y%
  }
  \anchor{S 1}{
    \pgf@process{\northeast}%
    \pgf@x=0\pgf@x%
    \pgf@y=-1\pgf@y%
  }
  \anchor{S 0}{
    \pgf@process{\northeast}%
    \pgf@x=0.666\pgf@x%
    \pgf@y=-1\pgf@y%
  }

  \anchor{output}{
   \pgf@process{\northeast}%
    \pgf@y=0.5\pgf@y%
  }

  % Draw the rectangle box and the port labels
  \backgroundpath{
    % Rectangle box
    \pgfpathrectanglecorners{\southwest}{\northeast}
 

    % Draw port labels
    \begingroup
    \tikzset{flip flop/port labels} % Use font from this style
    \tikz@textfont

    @\csname pgf@anchor@mux41@input 3\endcsname
    \pgftext[left,base,at={\pgfpoint{\pgf@x}{\pgf@y}},x=\pgfshapeinnerxsep]{\raisebox{-0.75ex}{11}}

    @\csname pgf@anchor@mux41@input 2\endcsname
    \pgftext[left,base,at={\pgfpoint{\pgf@x}{\pgf@y}},x=\pgfshapeinnerxsep]{\raisebox{-0.75ex}{10}}

    @\csname pgf@anchor@mux41@input 1\endcsname
    \pgftext[left,base,at={\pgfpoint{\pgf@x}{\pgf@y}},x=\pgfshapeinnerxsep]{\raisebox{-0.75ex}{01}}

    @\csname pgf@anchor@mux41@input 0\endcsname
    \pgftext[left,base,at={\pgfpoint{\pgf@x}{\pgf@y}},x=\pgfshapeinnerxsep]{\raisebox{-0.75ex}{00}}

    @\csname pgf@anchor@mux41@S 0\endcsname
    \pgftext[bottom,at={\pgfpoint{\pgf@x}{\pgf@y}},y=\pgfshapeinnerysep]{S$_0$}

    @\csname pgf@anchor@mux41@S 1\endcsname
    \pgftext[bottom,at={\pgfpoint{\pgf@x}{\pgf@y}},y=\pgfshapeinnerysep]{S$_1$}

    @\csname pgf@anchor@mux41@output\endcsname
    \pgftext[right,base,at={\pgfpoint{\pgf@x}{\pgf@y}},x=-\pgfshapeinnerxsep]{\raisebox{-.75ex}{Z}}

    \endgroup
  }
}

\tikzset{every mux41 node/.style={draw,minimum width=1.2cm,minimum 
height=1.7cm,inner sep=1mm,outer sep=0pt,cap=round,add font=\sffamily}}

\pgfdeclareshape{decoder38}{
  % The 'minimum width' and 'minimum height' keys, not the content, determine
  % the size
  \savedanchor\northeast{%
    \pgfmathsetlength\pgf@x{\pgfshapeminwidth}%
    \pgfmathsetlength\pgf@y{\pgfshapeminheight}%
    \pgf@x=0.5\pgf@x
    \pgf@y=0.5\pgf@y
  }
  % This is redundant, but makes some things easier:
  \savedanchor\southwest{%
    \pgfmathsetlength\pgf@x{\pgfshapeminwidth}%
    \pgfmathsetlength\pgf@y{\pgfshapeminheight}%
    \pgf@x=-0.5\pgf@x
    \pgf@y=-0.5\pgf@y
  }
  % Inherit from rectangle
  \inheritanchorborder[from=rectangle]

  % Define same anchor a normal rectangle has
  \anchor{center}{\pgfpointorigin}
  \anchor{north}{\northeast \pgf@x=0pt}
  \anchor{east}{\northeast \pgf@y=0pt}
  \anchor{south}{\southwest \pgf@x=0pt}
  \anchor{west}{\southwest \pgf@y=0pt}
  \anchor{north east}{\northeast}
  \anchor{north west}{\northeast \pgf@x=-\pgf@x}
  \anchor{south west}{\southwest}
  \anchor{south east}{\southwest \pgf@x=-\pgf@x}
  \anchor{text}{
    \pgfpointorigin
    \advance\pgf@x by -.5\wd\pgfnodeparttextbox%
    \advance\pgf@y by -.5\ht\pgfnodeparttextbox%
    \advance\pgf@y by +.5\dp\pgfnodeparttextbox%
  }

  % Define anchors for signal ports
  \anchor{output 7}{
    \pgf@process{\northeast}%
    \pgf@y=.87\pgf@y%
  }
  \anchor{output 6}{
    \pgf@process{\northeast}%
    \pgf@y=.66\pgf@y%
  }
  \anchor{output 5}{
    \pgf@process{\northeast}%
    \pgf@y=.45\pgf@y%
  }
  \anchor{output 4}{
    \pgf@process{\northeast}%
    \pgf@y=0.24\pgf@y%
  }
  \anchor{output 3}{
    \pgf@process{\northeast}%
    \pgf@y=0.03\pgf@y%
  }
  \anchor{output 2}{
    \pgf@process{\northeast}%
    \pgf@y=-0.18\pgf@y%
  }
  \anchor{output 1}{
    \pgf@process{\northeast}%
    \pgf@y=-0.39\pgf@y%
  }
  \anchor{output 0}{
    \pgf@process{\northeast}%
    \pgf@y=-0.6\pgf@y%
  }
  \anchor{S 2}{
    \pgf@process{\northeast}%
    \pgf@x=-0.75\pgf@x%
    \pgf@y=-1\pgf@y%
  }
  \anchor{S 1}{
    \pgf@process{\northeast}%
    \pgf@x=-0.25\pgf@x%
    \pgf@y=-1\pgf@y%
  }
  \anchor{S 0}{
    \pgf@process{\northeast}%
    \pgf@x=0.25\pgf@x%
    \pgf@y=-1\pgf@y%
  }

  % Draw the rectangle box and the port labels
  \backgroundpath{
    % Rectangle box
    \pgfpathrectanglecorners{\southwest}{\northeast}
 
    % Draw port labels
    \begingroup
    \tikzset{flip flop/port labels} % Use font from this style
    \tikz@textfont

    @\csname pgf@anchor@decoder38@output 7\endcsname
    \pgftext[right,base,at={\pgfpoint{\pgf@x}{\pgf@y}},x=-\pgfshapeinnerxsep]{\raisebox{-0.75ex}{111}}

    @\csname pgf@anchor@decoder38@output 6\endcsname
    \pgftext[right,base,at={\pgfpoint{\pgf@x}{\pgf@y}},x=-\pgfshapeinnerxsep]{\raisebox{-0.75ex}{110}}

    @\csname pgf@anchor@decoder38@output 5\endcsname
    \pgftext[right,base,at={\pgfpoint{\pgf@x}{\pgf@y}},x=-\pgfshapeinnerxsep]{\raisebox{-0.75ex}{101}}

    @\csname pgf@anchor@decoder38@output 4\endcsname
    \pgftext[right,base,at={\pgfpoint{\pgf@x}{\pgf@y}},x=-\pgfshapeinnerxsep]{\raisebox{-0.75ex}{100}}

    @\csname pgf@anchor@decoder38@output 3\endcsname
    \pgftext[right,base,at={\pgfpoint{\pgf@x}{\pgf@y}},x=-\pgfshapeinnerxsep]{\raisebox{-0.75ex}{011}}

    @\csname pgf@anchor@decoder38@output 2\endcsname
    \pgftext[right,base,at={\pgfpoint{\pgf@x}{\pgf@y}},x=-\pgfshapeinnerxsep]{\raisebox{-0.75ex}{010}}

    @\csname pgf@anchor@decoder38@output 1\endcsname
    \pgftext[right,base,at={\pgfpoint{\pgf@x}{\pgf@y}},x=-\pgfshapeinnerxsep]{\raisebox{-0.75ex}{001}}

    @\csname pgf@anchor@decoder38@output 0\endcsname
    \pgftext[right,base,at={\pgfpoint{\pgf@x}{\pgf@y}},x=-\pgfshapeinnerxsep]{\raisebox{-0.75ex}{000}}

    @\csname pgf@anchor@decoder38@S 0\endcsname
    \pgftext[bottom,at={\pgfpoint{\pgf@x}{\pgf@y}},y=\pgfshapeinnerysep]{S$_0$}

    @\csname pgf@anchor@decoder38@S 1\endcsname
    \pgftext[bottom,at={\pgfpoint{\pgf@x}{\pgf@y}},y=\pgfshapeinnerysep]{S$_1$}

    @\csname pgf@anchor@decoder38@S 2\endcsname
    \pgftext[bottom,at={\pgfpoint{\pgf@x}{\pgf@y}},y=\pgfshapeinnerysep]{S$_2$}

    \endgroup
  }
}

\tikzset{every decoder38 node/.style={draw,minimum width=1.8cm,minimum 
height=3cm,inner sep=1mm,outer sep=0pt,cap=round,add font=\sffamily}}

\pgfdeclareshape{decoder24}{
  % The 'minimum width' and 'minimum height' keys, not the content, determine
  % the size
  \savedanchor\northeast{%
    \pgfmathsetlength\pgf@x{\pgfshapeminwidth}%
    \pgfmathsetlength\pgf@y{\pgfshapeminheight}%
    \pgf@x=0.5\pgf@x
    \pgf@y=0.5\pgf@y
  }
  % This is redundant, but makes some things easier:
  \savedanchor\southwest{%
    \pgfmathsetlength\pgf@x{\pgfshapeminwidth}%
    \pgfmathsetlength\pgf@y{\pgfshapeminheight}%
    \pgf@x=-0.5\pgf@x
    \pgf@y=-0.5\pgf@y
  }
  % Inherit from rectangle
  \inheritanchorborder[from=rectangle]

  % Define same anchor a normal rectangle has
  \anchor{center}{\pgfpointorigin}
  \anchor{north}{\northeast \pgf@x=0pt}
  \anchor{east}{\northeast \pgf@y=0pt}
  \anchor{south}{\southwest \pgf@x=0pt}
  \anchor{west}{\southwest \pgf@y=0pt}
  \anchor{north east}{\northeast}
  \anchor{north west}{\northeast \pgf@x=-\pgf@x}
  \anchor{south west}{\southwest}
  \anchor{south east}{\southwest \pgf@x=-\pgf@x}
  \anchor{text}{
    \pgfpointorigin
    \advance\pgf@x by -.5\wd\pgfnodeparttextbox%
    \advance\pgf@y by -.5\ht\pgfnodeparttextbox%
    \advance\pgf@y by +.5\dp\pgfnodeparttextbox%
  }

  % Define anchors for signal ports
  \anchor{output 3}{
    \pgf@process{\northeast}%
    \pgf@y=0.8\pgf@y%
  }
  \anchor{output 2}{
    \pgf@process{\northeast}%
    \pgf@y=0.4\pgf@y%
  }
  \anchor{output 1}{
    \pgf@process{\northeast}%
    \pgf@y=0.0\pgf@y%
  }
  \anchor{output 0}{
    \pgf@process{\northeast}%
    \pgf@y=-0.4\pgf@y%
  }
  \anchor{S 1}{
    \pgf@process{\northeast}%
    \pgf@x=-0.666\pgf@x%
    \pgf@y=-1\pgf@y%
  }
  \anchor{S 0}{
    \pgf@process{\northeast}%
    \pgf@x=0\pgf@x%
    \pgf@y=-1\pgf@y%
  }


  % Draw the rectangle box and the port labels
  \backgroundpath{
    % Rectangle box
    \pgfpathrectanglecorners{\southwest}{\northeast}
 

    % Draw port labels
    \begingroup
    \tikzset{flip flop/port labels} % Use font from this style
    \tikz@textfont

    @\csname pgf@anchor@decoder24@output 3\endcsname
    \pgftext[right,base,at={\pgfpoint{\pgf@x}{\pgf@y}},x=-\pgfshapeinnerxsep]{\raisebox{-0.75ex}{11}}

    @\csname pgf@anchor@decoder24@output 2\endcsname
    \pgftext[right,base,at={\pgfpoint{\pgf@x}{\pgf@y}},x=-\pgfshapeinnerxsep]{\raisebox{-0.75ex}{10}}

    @\csname pgf@anchor@decoder24@output 1\endcsname
    \pgftext[right,base,at={\pgfpoint{\pgf@x}{\pgf@y}},x=-\pgfshapeinnerxsep]{\raisebox{-0.75ex}{01}}

    @\csname pgf@anchor@decoder24@output 0\endcsname
    \pgftext[right,base,at={\pgfpoint{\pgf@x}{\pgf@y}},x=-\pgfshapeinnerxsep]{\raisebox{-0.75ex}{00}}

    @\csname pgf@anchor@decoder24@S 0\endcsname
    \pgftext[bottom,at={\pgfpoint{\pgf@x}{\pgf@y}},y=\pgfshapeinnerysep]{S$_0$}

    @\csname pgf@anchor@decoder24@S 1\endcsname
    \pgftext[bottom,at={\pgfpoint{\pgf@x}{\pgf@y}},y=\pgfshapeinnerysep]{S$_1$}

    \endgroup
  }
}

\tikzset{every decoder24 node/.style={draw,minimum width=1.5cm,minimum 
height=1.7cm,inner sep=1mm,outer sep=0pt,cap=round,add font=\sffamily}}


\makeatother

%close label
\tikzstyle{clabel}=[label={[inner sep=0pt,label distance=1pt] #1}]

\tikzstyle{asm}=[draw,thick,fill=blue!20, minimum width=1.5cm, inner
    sep=2pt,align=center,on chain,minimum height=7mm]
\tikzstyle{state-output-list} =[asm,rectangle]
\tikzstyle{input-qualifier-expression}=[asm,shape=chamfered rectangle]
\tikzstyle{conditional-output-list} = [asm,shape=rounded rectangle]
\tikzstyle{statebox}=[state-output-list,label={[inner
    sep=0pt,label distance=1pt,font=\footnotesize] 45:#1}]

\tikzstyle{decision} = [asm, diamond, minimum height=1cm]
\tikzstyle{con} =[coordinate,on chain,join=by {- }]

\tikzstyle{F}=[clabel= #1:F]
\tikzstyle{T}=[clabel= #1:T]
\tikzstyle{block} = [rectangle, draw, fill=blue!20, 
    text width=5em, text centered,minimum height=10mm]
\tikzstyle{line} = [draw, thick, ->]
\tikzstyle{cloud} = [draw, ellipse,fill=red!20, node distance=3cm,
    minimum height=2em]

\tikzstyle{state-block}=[rectangle,draw,gray,ultra thick]


% styes and macros for timing diagrams
\tikzstyle{trace} =[black,thick,yscale=0.5,line join=round]
\tikzstyle{axis} = [blue!50]


\newcommand{\axis}[2]{
\foreach \y/\l in {#1} {
    \draw [axis] (-0.2,\y) edge [-latex] (#2+1,\y) 
          (0,\y) edge [-latex] ++(up:0.7)
          (-0.1,\y+0.3) node [black,anchor=east] {\l};
    \path (0,\y) coordinate (t\y) {};
    \foreach \x in {0,...,#2}
     \draw [axis] (\x+0.5,\y-0.1)--(\x+0.5,\y);
   }
   \foreach \x in {1,...,#2}
     \draw [trace,xshift=\x cm,thin,black]
       (-1,0)--(-0.5,0)  edge [very thick,-latex] (-0.5,1) (-0.5,1) --(0,1) -- (0,0);
}
